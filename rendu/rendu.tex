\documentclass[10pt,a4paper]{article}
\usepackage[utf8]{inputenc}
\usepackage[french]{babel}
\usepackage[T1]{fontenc}
\usepackage{amsmath}
\usepackage{amsfonts}
\usepackage{amssymb}
\usepackage{graphicx}
\usepackage[left=2cm,right=2cm,top=2cm,bottom=2cm]{geometry}
\usepackage{inconsolata}

\usepackage{listings}
\usepackage{color}

\definecolor{dkgreen}{rgb}{0,0.6,0}
\definecolor{gray}{rgb}{0.5,0.5,0.5}
\definecolor{mauve}{rgb}{0.58,0,0.82}

\lstset{frame=tb,
  language=c++,
  aboveskip=3mm,
  belowskip=3mm,
  showstringspaces=false,
  columns=flexible,
  basicstyle={\small\ttfamily},
  numbers=none,
  numberstyle=\tiny\color{gray},
  keywordstyle=\color{blue},
  commentstyle=\color{dkgreen},
  stringstyle=\color{mauve},
  breaklines=true,
  breakatwhitespace=true,
  tabsize=3
}

\author{Adrien GARBANI}
\title{Projet Bin Packing 1D}
\begin{document}
\maketitle

\section{Borne inférieure des jeux de données}

La borne inférieure du nombre de bin à utiliser est obtenue en divisant la somme de la taille de tous les objets par la taille d'un bin.

\begin{center}
  \begin{tabular}{ | c | c | c | c | }  \hline
    nom               & taille d'un bin & nombre d'items & nombre de bin \\ \hline
    binpack1d\_00.txt & 9               & 24             & 13            \\ \hline
    binpack1d\_01.txt & 150             & 250            & 99            \\ \hline
    binpack1d\_02.txt & 150             & 500            & 198           \\ \hline
    binpack1d\_03.txt & 1000            & 60             & 20            \\ \hline
    binpack1d\_04.txt & 1000            & 120            & 40            \\ \hline
    binpack1d\_05.txt & 1000            & 249            & 83            \\ \hline
    binpack1d\_06.txt & 1000            & 501            & 167           \\ \hline
    binpack1d\_11.txt & 100             & 50             & 25            \\ \hline
    binpack1d\_12.txt & 120             & 50             & 26            \\ \hline
    binpack1d\_13.txt & 120             & 500            & 252           \\ \hline
    binpack1d\_14.txt & 150             & 500            & 215           \\ \hline
    binpack1d\_21.txt & 10000           & 141            & 11            \\ \hline
    binpack1d\_31.txt & 1000            & 160            & 61            \\ \hline
  \end{tabular}
\end{center}

Résultats obtenus avec la commande : \texttt{./Project\_opti.exe question 1}.

\section{Résolution avec FirstFitDecreasing}

L'implémentation peut être trouvé dans la fonction \texttt{FileData::first\_fit\_decreasing()} dans le fichier \texttt{src/FileData.cpp}.

\begin{center}
  \begin{tabular}{ | c | c | c | }  \hline
    nom               & borne inférieure & FirstFitDecreasing \\ \hline
    binpack1d\_00.txt & 13               & 13                 \\ \hline
    binpack1d\_01.txt & 99               & 100                \\ \hline
    binpack1d\_02.txt & 198              & 201                \\ \hline
    binpack1d\_03.txt & 20               & 23                 \\ \hline
    binpack1d\_04.txt & 40               & 45                 \\ \hline
    binpack1d\_05.txt & 83               & 94                 \\ \hline
    binpack1d\_06.txt & 167              & 190                \\ \hline
    binpack1d\_11.txt & 25               & 25                 \\ \hline
    binpack1d\_12.txt & 26               & 29                 \\ \hline
    binpack1d\_13.txt & 252              & 258                \\ \hline
    binpack1d\_14.txt & 215              & 220                \\ \hline
    binpack1d\_21.txt & 11               & 12                 \\ \hline
    binpack1d\_31.txt & 61               & 62                 \\ \hline
  \end{tabular}
\end{center}

Résultats obtenus avec la commande : \texttt{./Project\_opti.exe question 2}.

\pagebreak
\section{Résolution linéaire}
\subsection{Solution de binpack1d\_00.txt}

La résolution linéaire donne un résultat optimal à 13 bins en environ $38ms$.
Le contenu de chaque bin étant :

\begin{center}
  \begin{tabular}{ | c | c | }  \hline
    items & total \\ \hline
    6     & 6     \\ \hline
    6,3   & 9     \\ \hline
    5,3   & 8     \\ \hline
    5,4   & 9     \\ \hline
    5,2,2 & 9     \\ \hline
    4,5   & 9     \\ \hline
    4,5   & 9     \\ \hline
    4,4   & 8     \\ \hline
    2,7   & 9     \\ \hline
    2,7   & 9     \\ \hline
    5,4   & 9     \\ \hline
    8     & 8     \\ \hline
    8     & 8     \\ \hline
   \end{tabular}
\end{center}

Résultats obtenus avec la commande : \\ \texttt{./Project\_opti.exe question 3 file resources/binpack1d\_00.txt verbose}.

\subsection{Limite de la résolution linéaire}

\section{Générateurs aléatoires}

Les générateurs de solution aléatoires peuvent être trouvés dans les fonctions \texttt{FileData::solve\_simple()} et \texttt{FileData::first\_fit\_random()} dans le fichier \texttt{src/FileData.cpp}. 

Leur résultats peuvent être obtenus avec la commande : \\ \texttt{./Project\_opti.exe question 4 file resources/binpack1d\_00.txt verbose}.

\section{Opérateurs de voisinage}



\if 0

\fi

\end{document}

\begin{lstlisting}
// ...
\end{lstlisting}
\begin{center}
  \begin{tabular}{ | c | c | c | }  \hline
    a & b & c \\ \hline
    1 & 2 & 3 \\ \hline
   \end{tabular}
\end{center}
